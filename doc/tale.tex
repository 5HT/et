% copyright (c) 2016 Groupoid Infinity
\documentclass[11pt,oneside]{article}

%% Copyright (c) 2010 Synrc Research Center

\usepackage{ifthen}
\usepackage[english,russian]{babel}
\usepackage{bussproofs}
\usepackage{tabstackengine}
\usepackage{graphicx}
\usepackage{cite}
\usepackage{hyperref}
\usepackage[utf8]{inputenc}
\usepackage[T1]{fontenc}
\usepackage{listingsutf8}
\usepackage{moreverb}
\usepackage{listings}
\usepackage[none]{hyphenat}
\usepackage{caption}
\usepackage[usenames,dvipsnames]{color}
\usepackage[hmarginratio=2:3]{geometry}
\usepackage{inconsolata}
\usepackage{amssymb}
\usepackage{mathtools}

\hyphenation{framework nitrogen javascript facebook}

% include image for HeVeA and LaTeX

\makeatletter
\def\@seccntformat#1{\llap{\csname the#1\endcsname\quad}}
\makeatother

\newcommand{\includeimage}[2]
{\ifhevea
    {\imgsrc{#1}}
\else{
    \begin{figure}[h!]
    \centering
    \includegraphics[width=\textwidth]{#1}
    \caption{#2}
    \end{figure}}
\fi}

\lstset{
    backgroundcolor=\color{white},
    keywordstyle=\color{blue},
    basicstyle=\bf\ttfamily\footnotesize,
    columns=fixed}

\headsep = 0cm
\voffset = 0cm
\hoffset = 1cm
\topmargin = 0cm
\textwidth = 15cm
\textheight = 22cm
\footskip = 1.3cm
\parindent = 0cm

\hyphenpenalty=5000
  \tolerance=1000

\newcommand{\sign}[1]{%      
  \begin{tabular}[t]{@{}l@{}}
  \makebox[1.5in]{\dotfill}\\
  \strut#1\strut
  \end{tabular}%
}
\newcommand{\Date}{%
  \begin{tabular}[t]{@{}p{1.5in}@{}}
  \\[-1ex]
  \strut Date: \dotfill\strut
  \end{tabular}%
}

\addto\captionenglish{\renewcommand*{\bibname}{Literaturliste}}

\lstset{
 inputencoding=utf8,
extendedchars=true
}

\usepackage[english]{babel}
\begin{document}

\thispagestyle{empty}
\begin{center}
\vspace{3cm}
    \vspace{3cm}   {\Large \bf The Long Tale of Implementing CIC over CoC\\}\par
    \vspace{0.3cm} {\Large Technical Article\par}
    \vspace{0.3cm} {\Large Paul Lyutko, Groupoid Infinity\par}
    \vspace{4cm}   {\Large Kyiv 2016}
\end{center}

\newpage
\vspace{2cm}
\tableofcontents

\newpage
\section{Introduction: The Why}

Ladies and gentlemen! Let us start from the beginning.
In the beginning there were quantifiers and lambdas binding variables in the terms.
This is what any Pure Type System is built from.
It can also be seen as a category made from types (as objects) and functions (as morphisms).
If we want we can extend this category with any constructions
we need but does it really need to be extended?
What if it already contains all kinds of structures we know?
Hereby we claim --- yes, it contains.
The structures we seek are inductive types given by their constructors and eliminators,
and coinductive types, and higher inductive types, and also
plenty of types formed by appropriate adjoint functors.

Such approach allows us to implement a full-featured language (EXE compiler)
with dependent types atop of a small logical core of pure lambda calculus (OM compiler).
You know the LISP language uses "code as data" metaphor, but we use "data as code" insteed!

It this text we discuss an encoding. An encoding is a model of one type-theory-like formal
system in another type-theory-like formal system. The encoding we use implements axioms
of Calculus of Inductive Construction (CIC), the type theory of inductive types,
in Calculus of Construction, a Pure Type System (PTS).
We do it by mapping types not to types but to structures known
as setoids. Setoids provide with a well-behaved equality on our types and on the whole category.
We also utilize the fact that this category has limits of functors going into it,
a categorical notion refining dependent function type from PTS.

Attempts to explore encodings are not new. An historical example is the
Church encoding for natural numbers and its generalizations [Berarducci].
Nowadays there are pragmatically charged compilers using them [Gonzalez].
We introduce an encoding capable of implementing dependent eliminator a.k.a.
induction principle built by refining the Berarducci encoding with categorical limits.
Attention: we do not extend the preexistent category of types,
we just discover all required kinds of structures in the category of setoids.
Here it goes. Enjoy!

\newpage
\section{Steps: The How}

In this section we describe all the constructions used.
Explanations are intended to be read by software engineers (really?).
If you are a mathematician, just read the phrase
"application of General Adjoint Functor Theorem to type theory" and meditate on it.

\subsection{Forall and Lambda}

If you are used to lambda expressions in functional programming
you will find the following familiar.

The mathematical language we use is called Calculus of Constructions.

It is a typed programming language with 5 constructions: variable names, anonymous functions,
type declarations, function calls/argument applications and predefined types.

Variables can be only defined as parameters to functions or parameters to type declarations.
There are no stand-alone variables, every variable is a parameter somewhere. Moreover,
functions and types always have one argument. There are no functions or types without arguments,
and they cannot have more than one.

Also, there are no "normal" data: all the data is represented by functions.

From now on, we will use mathematical names for the constructions of the language.

Anonymous functions are called lambda-abstractions or simply lambdas,
and are denoted by  $\lambda$.

Type declarations are denoted by $\forall$ and are called function types
(remember, everything is a function). Sometimes $\forall$ doesn't use the variable it declares,
so we will use $\rightarrow$ as a shortcut.

Predefined types are called universes. In original Calculus of Constructions
there were only two universes, an asterisk and a square. But our variation has
many, and for each universe there is a "larger" universe, so they form an infinite sequence. We will use *0, *1, *2 and so on.
As very large universes are rarely needed, later we will introduce special names for few first universes.

As both types and functions have formal parameters, there is a way to pass the actual parameters. Unlike most normal programming languages, when passing value arguments are denoted by foo(bar) and passing type arguments is denoted by foo<bar>, in
CoC they are both written as foo bar, and foo bar baz means (foo bar) baz, i.e. foo bar "returns" something that can be "called" again.

For those mathematically inclined, Calculus of Construction is a Pure Type System, and Pure Type Systems are typed lambda-calculi.

Its language consists of variable references,
functions (known as lambda abstractions,
denoted by $\lambda$),
function types (denoted by $\forall$)
and applications of an argument to a function
(denoted by juxtaposition).
There are also universes (types of types).
There are no others expressions!
We will call the expressions of our language  {\em terms}.

We are not going to introduce them formally by so-called
"rules of inference" notation. Instead, we will show generic examples.

\paragraph{Typing}
Notation $a : A$ means the term $a$ is of type $A$.

\paragraph{Abstraction}
If there is a term $y : Y$ containing variable $x : X$,
then there is a new term
$(\lambda x : X) \rightarrow y : (\forall x : X) \rightarrow Y$.
The former is the term of dependent function, the latter is its type.
Also we write $A \rightarrow B$ for shorthand of
$(\forall a : A) \rightarrow B$ if the variable $a$
does not appear in term $B$
(it is the well-known type of non-dependent functions).

\paragraph{Application}
If $f : (\forall x : X)\rightarrow Y$ and $v : X$  then $(f v) : Y[v/x]$.
Also we assume $((\lambda x : X)\rightarrow y) v$
to be equal to $y[v/x]$.
Here $f[v/x]$ means the substitution of the term $v$ in place of all
occurencies of the variable $x$ in the term $f$.

\paragraph{Universes}
If $a : A$ then $A : *$.
Here we denote "type of types" as $*$ symbol.
These are also called universes.
But what is the type of $*$ itself? Can we let it be just $*$?
No, it appears that $* : *$ causes paradoxes, so
we have to consider different universes.
We can consider just one "superuniverse" $BOX$
(then the whole system will be exactly CoC),
or else we can provide an infinite hierarchy of similar universes
of level $n$, denoted $*{n}$,
with the next universe containing the previous one,
and this is more general PTS.

\paragraph{Predicativity}
If we have different universes, we should ask what is the universe
of the type of functions between given types.
Let $M : *{m}$ and $N : *{n}$ and assume $(M \rightarrow N) : *{r} $.
There are two simple answers:
$r = max (m, n)$ (called the predicative hierarchy) and
$r = n$ (called the impredicative hierarchy).

\subsection{Logics}
In CoC/PTS the universes are first-class citizens,
and they can be used freely in functions and in function types.
Such usage allows not only dependent types, e.g. "type of lists of the
given length", but also logical propositions.
In the "types as propositions" approach we consider the empty type
as the logical FALSE, and any inhabited type as the logical TRUE,
and inhabitants of types are to be seen as evidences or proofs.
Therefore we are able to state requirements against elements of a type
(predicates) by forming appropriate dependent types
and to check (verify) them making terms of that type.
For further explanation of so-called Curry-Howard correspondence
see [111].
We just note that there is a complex bussiness of
describing the whole world in the language of dependent types
mirroring older {\em predicate calculus} known from the XIX century.

\paragraph{Important}
Any well-typed term in PTS is simultaneously a proof of the theorem stated by its type.

\subsection{Contexts: Curried Records}

A context is a notion in type theory meaning a sequence of types of variables
needed to formally define a constraint on binding of all free variable in a term.
For example, $( T : Type, x : T )$ declares a context including two variable with
the type of the second variable being dependent on the value of the first one.

We can define a notion of a function between two contexts.
For contexts $A = (A_1, A_2, ...)$ and $B = (B_1, B_2, ...)$
the type of such functions $A \rightarrow B$ is a context declaring
function variables $(f_1 : A \rightarrow B_1, f_2 : A \rightarrow B_2, ...)$.
All functions should receive all variables from the context $A$ and
emit just one variable from the context $B$.

For the given context there is a notion of {\em tuple type}
as a type of all sequences of terms of the given types.
We need to use tuples (a.k.a. records) but PTSs have not tuples.

So this is the first layer of our encoding.

We simply shift our focus from types to context and functions between them.
We can simply consider context as tuple type.
The pragmatic viewpoint allows such a shift because we are able to use
types and terms in the high-level language being compiled to contexts and and sequences of terms in the low-level one.

Also there is a well-known phenomenon in type theory --- a curring.
A curring means that function with a tuple argument is
equivalent to a function with a number of arguments.
Here we have curring by desing.

\subsection{Setoids, Mappings}

\subsection{Categories, Functors}

\subsection{Initial and Terminal Objects}

\subsection{Inductive types and their eliminators}

\subsection{Limits and Colimits}

\subsection{Adjoint Functor Theorem}

\subsection{Categories of Dialgebras}

\subsection{Limits of Setoids}

\subsection{Creating Limits of Dialgebras}

\subsection{Simple Ornaments and Polymonial Functors}

\subsection{Getting Induction from Recursion}



\section{Examples: The Pragmatics}
\subsection{Basic Algebraic dataTypes}
\subsection{The List dataType}

\section{Advanced: There and Back Again}
\subsection{Free monad}
\subsection{Free algebraic structures}
\subsection{Existential types}
\subsection{Colimits}
\subsection{Coinductive types}
\subsection{Dependent inductive types}
\subsection{The Synthetic Universe}

\section{HoTT: The beyond}
\subsection{Infinity-Groupoids}
\subsection{Truncation}
\subsection{Higher Inductive types}
\subsection{Univalence}

\newpage
\section{References}
\begin{thebibliography}{9}

\bibitem{henk0}      Henk Barendregt \textit{The Lambda Calculus. Its syntax and semantics} 1981
\bibitem{henk1}      Henk Barendregt \textit{Lambda Calculus With Types} 2010
\bibitem{henk}       Erik Meijer, Simon Peyton Jones \textit{Henk: a typed intermediate language} 1984
\bibitem{lof}        Per Martin-Löf \textit{Intuitionistic Type Theory} 1984
\bibitem{berarducci} ????
\bibitem{gonzalez} ????

\end{thebibliography}
\end{document}
