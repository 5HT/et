% copyright (c) 2016 Groupoid Infinity
\documentclass[11pt,oneside]{article}

%% Copyright (c) 2010 Synrc Research Center

\usepackage{ifthen}
\usepackage[english,russian]{babel}
\usepackage{bussproofs}
\usepackage{tabstackengine}
\usepackage{graphicx}
\usepackage{cite}
\usepackage{hyperref}
\usepackage[utf8]{inputenc}
\usepackage[T1]{fontenc}
\usepackage{listingsutf8}
\usepackage{moreverb}
\usepackage{listings}
\usepackage[none]{hyphenat}
\usepackage{caption}
\usepackage[usenames,dvipsnames]{color}
\usepackage[hmarginratio=2:3]{geometry}
\usepackage{inconsolata}
\usepackage{amssymb}
\usepackage{mathtools}

\hyphenation{framework nitrogen javascript facebook}

% include image for HeVeA and LaTeX

\makeatletter
\def\@seccntformat#1{\llap{\csname the#1\endcsname\quad}}
\makeatother

\newcommand{\includeimage}[2]
{\ifhevea
    {\imgsrc{#1}}
\else{
    \begin{figure}[h!]
    \centering
    \includegraphics[width=\textwidth]{#1}
    \caption{#2}
    \end{figure}}
\fi}

\lstset{
    backgroundcolor=\color{white},
    keywordstyle=\color{blue},
    basicstyle=\bf\ttfamily\footnotesize,
    columns=fixed}

\headsep = 0cm
\voffset = 0cm
\hoffset = 1cm
\topmargin = 0cm
\textwidth = 15cm
\textheight = 22cm
\footskip = 1.3cm
\parindent = 0cm

\hyphenpenalty=5000
  \tolerance=1000

\newcommand{\sign}[1]{%      
  \begin{tabular}[t]{@{}l@{}}
  \makebox[1.5in]{\dotfill}\\
  \strut#1\strut
  \end{tabular}%
}
\newcommand{\Date}{%
  \begin{tabular}[t]{@{}p{1.5in}@{}}
  \\[-1ex]
  \strut Date: \dotfill\strut
  \end{tabular}%
}

\addto\captionenglish{\renewcommand*{\bibname}{Literaturliste}}

\lstset{
 inputencoding=utf8,
extendedchars=true
}

\begin{document}

\thispagestyle{empty}
\begin{center}
\vspace{3cm}
    \vspace{3cm}   {\Large \bf The Long Tale of Implementing CIC over CoC\\}\par
    \vspace{0.3cm} {\Large Technical Article\par}
    \vspace{0.3cm} {\Large Paul Lyutko, Groupoid Infinity\par}
    \vspace{4cm}   {\Large Kyiv 2016}
\end{center}

\newpage
\vspace{2cm}
\tableofcontents

\newpage
\section{Introduction: The Why}

Ladies and gentlemen! Let us start from the beginning.
In the beginning there were quantifiers and lambdas binding variables in the terms.
This is what any Pure Type System is built from.
It can also be seen as a category made from types (as objects) and functions (as morphisms).
If we want we can extend this category with any constructions
we need but does it really need to be extended?
What if it already contains all kinds of structures we know?
Hereby we claim --- yes, it contains.
The structures we seek are inductive types given by their constructors and eliminators,
and coinductive types, and higher inductive types, and also
plenty of types formed by appropriate adjoint functors.

Such approach allows us to implement a full-featured language (EXE compiler)
with dependent types atop of a small logical core of pure lambda calculus (OM compiler).
You know the LISP language uses "code as data" metaphor, but we use "data as code" insteed!

It this text we discuss an encoding. An encoding is a model of one type-theory-like formal
system in another type-theory-like formal system. The encoding we use implements axioms
of Calculus of Inductive Construction (CIC), the type theory of inductive types,
in Calculus of Construction, a Pure Type System (PTS).
We do it by mapping types not to types but to structures known
as setoids. Setoids provide with a well-behaved equality on our types and on the whole category.
We also utilize the fact that this category has limits of functors going into it,
a categorical notion refining dependent function type from PTS.

Attempts to explore encodings are not new. An historical example is the
Church encoding for natural numbers and its generalizations [Berarducci].
Nowadays there are pragmatically charged compilers using them [Gonzalez].
We introduce an encoding capable of implementing dependent eliminator a.k.a.
induction principle built by refining the Berarducci encoding with categorical limits.
Attention: we do not extend the preexistent category of types,
we just discover all required kinds of structures in the category of setoids.
Here it goes. Enjoy!

\section{Steps: The How}
\subsection{Forall and Lambda}
\subsection{Contexts: Curried Records}
\subsection{Setoids, Mappings}
\subsection{Categories, Functors}
\subsection{Initial and Terminal Objects}
\subsection{Inductive types and their eliminators}
\subsection{Limits and Colimits}
\subsection{Adjoint Functor Theorem}
\subsection{Categories of Dialgebras}
\subsection{Limits of Setoids}
\subsection{Creating Limits of Dialgebras}
\subsection{Simple Ornaments and Polymonial Functors}
\subsection{Getting Induction from Recursion}

\section{Examples: The Pragmatics}
\subsection{Basic Algebraic dataTypes}
\subsection{The List dataType}

\section{Advanced: There and Back Again}
\subsection{Free monad}
\subsection{Free algebraic structures}
\subsection{Existential types}
\subsection{Colimits}
\subsection{Coinductive types}
\subsection{Dependent inductive types}
\subsection{The Synthetic Universe}

\section{HoTT: The beyond}
\subsection{Infinity-Groupoids}
\subsection{Truncation}
\subsection{Higher Inductive types}
\subsection{Univalence}

\newpage
\section{References}
\begin{thebibliography}{9}

\bibitem{henk0}      Henk Barendregt \textit{The Lambda Calculus. Its syntax and semantics} 1981
\bibitem{henk1}      Henk Barendregt \textit{Lambda Calculus With Types} 2010
\bibitem{henk}       Erik Meijer, Simon Peyton Jones \textit{Henk: a typed intermediate language} 1984
\bibitem{lof}        Per Martin-Löf \textit{Intuitionistic Type Theory} 1984
\bibitem{berarducci} ????
\bibitem{gonzalez} ????

\end{thebibliography}
\end{document}
